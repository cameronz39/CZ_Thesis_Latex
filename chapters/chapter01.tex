\chapter{Introduction}

\section{Motivation}
Spacecraft dynamics simulators are used extensively in the development and testing of all modern space systems~\cite{schwartz_historical_2003}. The ability of space organizations to test both their flight hardware and software before launch is a crucial part in reducing on-orbit
failures. The primary goal of all spacecraft dynamics simulators is to recreate the frictionless, torque-free environment of space. The approaches to this goal vary slightly as will be discussed in Chapter 2, but the primary method mounts the spacecraft hardware to a spherical air-bearing, allowing the platform to freely rotate about its yaw axis, and within some limited range about it's pitch and roll axes. The high pressure air between the spherical mount and platform ensures rotation with near-zero friction~\cite{huang_characterizing_2022}.

While an air-bearing helps guarantee frictionless rotation, any distance between the platform's center of mass and it's center of rotation will introduce a torque due to gravity that is not seen in space. Simulator's are able to shift their center of mass by changing the position of sliding masses onboard~\cite{kim_automatic_2009}. The end goal is to adjust the mass blocks' positions such that the mass distribution of the platform changes, and the distance between the platform's center of mass and center of rotation is minized. These masses may be adjusted by hand with visual inspection (referred to as manual mass balancing), or they may be precisely controlled using a preset balancing algorithm and linear actuators (referred to as automatic mass balancing). The detailed requirements of a mass balancing system will be discussed in \Cref{chap:background}, but in general, all simulators must perform some form of balancing this balancing process should both minimize the torque due to gravity and run as fast as possible.



\section{Previous Work}

Research laboratories and universities take part in building their own simulators for educational purposes and as a way to test novel control algorithms and hardware. The Cal Poly Spacecraft Attitude Dynamics Simulator (SADS) is part of this category and has been developed by students and faculty over the years. The mass balancing system on the SADS has gone under numerous iterations which are summarized in \Cref{table:sads_history}.

\begin{table}[h!]
\caption{Summary of previous work on the Cal Poly Spacecraft Attitude Dynamics Simulator (SADS)}
\label{table:sads_history}
\centering
\renewcommand{\arraystretch}{1.4} % row height

\begin{tabularx}{\textwidth}{
    >{\raggedright\arraybackslash}p{4cm}   % Author/Year
    >{\raggedright\arraybackslash}p{5cm}   % Title
    >{\raggedright\arraybackslash}X}       % Notes
\hline
\textbf{Author / Year} & \textbf{Title} & \textbf{Notes} \\
\hline
Mittelsteadt, Mehiel [2007] & 
Cal Poly Spacecraft Attitude Dynamics Simulator: CP/SADS & 
First iteration of the Cal Poly SADS, purely manual balancing \\
[2.0em]

Silva [2010] & 
Applied System Identification for a Four Wheel Reaction Wheel Platform & 
Introduced least squared estimation method to estimate center of mass \\
[2.0em]

Dam [2014] & 
Applied Mass Properties Identification Method to the Cal Poly's Spacecraft Simulator & 
Used least squared estimation method with significantly improved measurement system \\
[2.0em]

Gilman [2024] & 
Automatic Mass Balancing of a Spacecraft Attitude Dynamics Simulator with Six Sliding Masses & 
Completely replaced mass balancing hardware, only performed hardware-in-the-loop tests \\
\hline
\end{tabularx}
\end{table}

The platform was first introduced in 2007~\cite{mittelsteadt_cal_2007} and featured a set mass blocks that slid along slots cut out from the structure. These blocks had to be adjusted by hand, and adjustments were made by observing the platforms inbalance visually and making incremental changes.

In 2010, Silva attempted to improve on this method by using a least-squares estimation algorithm to estimate the center of mass. This requires taking body rate when the platform is tumbling under the influence of gravity. Unfortunately the measurement system at the time only could provide low-rate gyroscope data which significantly hindered the balancing results. Dam followed up on this in 2014 using a navigation-grade IMU to provide high-quality body rate data at 20 Hz. This significantly improved balancing results, with the final estimated center of mass after balancing being on the order of $10^{-3}$ meters.

Beginning in 2024? an effort was made to modernize the hardware on the SADS, which included the MBS.\@ In 2024 Gilman worked on a set of custom linear actuators that could finely control the positions of sliding masses onboard. Due to a lack of a finished flight computer and measurement system, the actuators could only be tested in hardware-in-the-loop conditions. How these various mass balancing implementations compare against those found in literature will be discussed in depth in \Cref{chap:background}. 


\section{Thesis Objectives}

The primary goal of this thesis is to design, integrate and test a mass balancing system that can be run quickly and achieves similar or better balancing results than previous iterations - namely the hardware results achieved by Dam in 2014. To achieve this, work will also need to be done on the SADS' overall design that will 

- Compare the results of various balancing methods found in literature on the same platform, provide a clear comparison to theortical and real results, provide practical insights on MBS design and implementation (there is an apprent gap for this in literature, with an abundance of papers providing theoretical approaches, but few providing experimental results, and fewer discussing the details of said experiments)

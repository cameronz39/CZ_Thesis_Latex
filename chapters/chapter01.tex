\chapter{Introduction}

\section{Motivation}
Spacecraft dynamics simulators are used extensively in the development and testing of all modern space systems [1]. The ability of space organizations to test both their flight hardware and software before launch is a crucial part in reducing on-orbit
failures. The primary goal of all spacecraft dynamics simulators is to recreate the frictionless, torque-free environment of space. The approaches to this goal vary slightly as will be discussed in Chapter 2, but the primary method mounts the spacecraft hardware to a spherical air-bearing, allowing the platform to freely rotate about its yaw axis, and within some limited range about it's pitch and roll axes. The high pressure air between the spherical mount and platform ensures rotation with near-zero friction [2].

\section{Previous Work}

\section{Thesis Objectives}

\iffalse 
- Create a balancing system that can be run quickly and achieves similar or better balancing results than previous iterations
- Advance work on the SADS overall design that will be necessary for other subsystems, namely a measurement system and early iteration of a flight computer, track lessons learn
- Compare the results of various balancing methods found in literature on the same platform, provide a clear comparison to theortical and real results, provide practical insights on MBS design and implementation (there is an apprent gap for this in literature, with an abundance of papers providing theoretical approaches, but few providing experimental results, and fewer discussing the details of said experiments)
\fi 
\chapter{Results and Discussion}



\section{Underactuated PID Control}

The underactuated PID controller is first implemented in simulation according \Cref{sec:sim_setup}. \Cref{fig:PID_sim_results} shows key system states during the balancing procedure. The platform's roll initially drops to $-17^{\circ}$ due to gravity, but the proportional action in the action in the controller quickly compensates as seen by the rapid spike in $\Delta\,d_y$. After about 45 seconds, the platform's roll, pitch, and horizontal body rates converge to 0. The sliding mass positions additionally settle to a constant value, indicating the gravitional torque has been driven towards 0. The simulation also verifies that the roll and pitch travel limits of the simulator ($\pm20^{\circ}$) have not been violated, and that the maximum speed and travel limits of the sliding masses have not been violated, even under a relatively large inital imbalance.

\begin{figure}[!ht]
    \centering
    \includegraphics[width=\linewidth]{plots/PID_sim_results.pdf}
    \caption{Simulated balancing results using PID control}
    \label{fig:PID_sim_results}
\end{figure}

\begin{figure}[!ht]
    \centering
    \includegraphics[width=0.9\linewidth]{plots/PID_sim_integral_error.png}
    \caption{Projected integral error during a simulated PID run}
    \label{fig:PID_sim_integral_error}
\end{figure}

From \Cref{fig:PID_sim_results} alone it can be inferred the controller and has horizontally balanced the platform. However, to further validate the performance of the balancing method, various values from the truth model are plotted. \Cref{fig:PID_sim_integral_error} shows the projected integral error of the controller. It is clear the integral action does indeed learn and compensate for the initial gravitional torque $\bm{\tau}_{g,0}$. The effects of the simulated controller can also be seen, with the 15 Hz discrete time integrator causing noise at steady state.

\begin{figure}[!ht]
    \centering
    \includegraphics[width=0.9\linewidth]{plots/PID_sim_oscillations.pdf}
    \caption{Simulated PID scillatations about $\bm{r}=0$ at steady state}
    \label{fig:PID_sim_oscillations}
\end{figure}

\Cref{fig:PID_sim_oscillations} shows the true value of $r_x$ and $r_y$ at steady state. The values begin to oscillate with a fairly constant amplitude about the desired values of $r_x = r_y = 0$. A variety of factors here prevent further convergence. The discretization of the controller, noise in the sensor readings, and lag in the control loop all limit the performance of the system. In a real hardware test, the operator will end the balancing procedure once the sliding mass positions have converged. It can be inferred then that the final value $r_x$ and $r_y$ will land at a random position along the two curves in \Cref{fig:PID_sim_oscillations}. The ampltiude of these oscillations places a bound on the final balancing results. In this case, $r_x$ and $r_y$ are bounded by $\pm4\times10^{-5}m$, so in the worst case, the simulated PID balancing procedure has still reduced the horizontal inbalance by two orders of magnitude.

Next the experimental PID results on the SADS are considered. \Cref{fig:PID_hardware_results} shows the same system states during the balancing procedure. The system qualitatively behaves in a similar manner to the simulated results. The platform's roll, pitch, and horizontal body rates converge to 0 and the sliding mass positions additionally converge towards a constant value. The values for all experimental system states are all generally smaller in magnitude compared to the simulated values from \Cref{fig:PID_sim_results}, indicating the initial imbalance was smaller.  

While a plot like \Cref{fig:PID_sim_oscillations} cannot be recreated in an experimental test, the oscillations of $\Delta\bm{d}$ in \Cref{fig:PID_hardware_results} can be converted into oscillations of $\Delta\bm{r}$ using \Cref{equation:delta_d_sol}. This leads to $\Delta\,r_x$ and $\Delta\,r_y$ being bounded by $\pm2.5\times10^{-5}m$, which is similar to simulated value of $\pm4\times10^{-5}m$. This suggests good alignment between the simulated system and the experimental system, however it is unknown if the oscillations in the experimental test are about the desired values of $r_x = r_y = 0$.



\begin{figure}[!ht]
    \centering
    \includegraphics[width=\linewidth]{plots/PID_hardware_results.pdf}
    \caption{Experimental balancing results using PID control}
    \label{fig:PID_hardware_results}
\end{figure}

\section{Underactuated Adaptive Control}

Next, the adaptive control method is implemented into the same simulation. Attempts to simulate this method however quickly created numerical instability within the simulation. To alleviate this, all sources of error discussed in \Cref{sec:sim_setup} are temporarily removed from the simulation which results in \Cref{fig:adaptive_sim_success}. The results here indicate the implemented control law works under the mathmatically ideal case. $\bm{\omega}_p$ converges to $\bm{0}$ and the sliding mass converge to the position that minimizes $r_x$ and $r_y$. However, the sliding mass velocites and rapid oscillations of the platform are not physically feasible. This behaviour is identical to the results of \cite{silva_filtering_2018} and \cite{hudson_dynamic_2019}, which both attempt to simulate the same adaptive controller. After many attempts to diagose the source of instability, it remains unclear what differs between the original implementation of this balancing method in \cite{chesi_automatic_2014}, and the implementation in \cite{silva_filtering_2018}, \cite{hudson_dynamic_2019}, and this work.

\begin{figure}[!ht]
    \centering
    \includegraphics[width=\linewidth]{plots/adaptive_sim_success.pdf}
    \caption{Simulated underactuated adaptive control results using a simplified model}
    \label{fig:adaptive_sim_success}
\end{figure}

\begin{figure}[!ht]
    \centering
    \includegraphics[width=\linewidth]{plots/adaptive_sim_failure.pdf}
    \caption{Simulated underactuated adaptive control results with sensor dyanmics and onboard filtering modeled}
\end{figure}

Given the relative ease of changing control laws in the Matlab/Simulink environment, the underactuated adaptive control method is still implemented on the SADS to experiment which is seen in \Cref{fig:adaptive_hardware_failure}. Even with exensive tuning of the control gains, the simulator is generally unstable. The roll of the simulator at 120 seconds peaks at 0.4 radians, which is just beyond the $20^{\circ}$ safety limit of the test setup. In fact, \Cref{fig:adaptive_hardware_failure} shows the longest of the attempts that all were manually ended due to the simulator leaving its roll and pitch travel limits. Given the success of the PID method which has the identical goal of horizontally balancing the simulator, it was eventually decided to leave further investigation into the adaptive method as future work.  



\begin{figure}[!ht]
    \centering
    \includegraphics[width=\linewidth]{plots/adaptive_hardware_failure.pdf}
    \caption{Experimental results using underactuated adaptive control}
    \label{fig:adaptive_hardware_failure}
\end{figure}


\section{Vertical Inbalance Estimation with Kalman Filtering}


\begin{figure}[!ht]
  \centering
  \begin{subfigure}[t]{0.47\textwidth}
    \includegraphics[width=\linewidth]{plots/UKF_comparison.pdf}
  \end{subfigure}\hfill
  \begin{subfigure}[t]{0.47\textwidth}
    \includegraphics[width=\linewidth]{plots/UKF_comparison_zoomed.pdf}
  \end{subfigure}
  \caption{Filter results over multiple iterations, zoomed in on right}
  \label{fig:UKF_comparison}
\end{figure}


With the underactuated PID controller successfully horizontally balancing the platform, the UKF is then used to estimate the vertical inbalance. The free tumbling body rates of the platform are used as the filter input, with simulated tumbling data being used first to verify convergence. \Cref{fig:UKF_hardware_results} shows the filters output and $1\sigma$ confidence interval, where $\sigma$ is computed from the covariance matrix as $\sigma=\sqrt{P_k(7,7)}$. The filter shows strong convergence as $r_z$ is adjusted between each run in simulation. 

After filter convergence was demonstrated in simulation, real tumbling data from the SADS is used as input. The filter output over time during two selected iterations are shown in \Cref{fig:UKF_hardware_results}. \Cref{fig:UKF_hardware_results_a} shows the very first iteration after PID balancing. The filter quickly converges on an estimate for $r_z$, and the corresponding $1\sigma$ window is again plotted to demonstrate the filter's confidence. \Cref{fig:UKF_hardware_results_b} shows the filter output just two iterations later, where $r_z$ simply oscillates about 0. While the simulated and experimental results show similar behaviour, the simulated filter confidence is significantly higher, potentially indicating the modeling in \Cref{sec:sim_setup} did not introduce enough error into the simulation. 

\begin{figure}[!ht]
  \centering
  \begin{subfigure}[t]{0.47\textwidth}
    \includegraphics[width=\linewidth]{plots/UKF_hardware_success.pdf}
    \caption{Iteration successfully converging}\label{fig:UKF_hardware_results_a}
  \end{subfigure}\hfill
  \begin{subfigure}[t]{0.47\textwidth}
    \includegraphics[width=\linewidth]{plots/UKF_hardware_failure.pdf}
    \caption{Iteration failing to converge}\label{fig:UKF_hardware_results_b}
  \end{subfigure}
  \caption{Experimental filter output during vertical inbalance estimation}
  \label{fig:UKF_hardware_results}
\end{figure}

In general as the true value of $r_z$ shrinks, the filter's ability to estimate $r_z$ also decreases. This is further demonstrated in \Cref{fig:UKF_comparison}, where the estimated values clearly tend towards 0, but the magnitude of the confidence interval starts to grow, even including positive values for $r_z$. This is important to consider when vertically balancing the simulator. During experimental tests it was observed the simulator could begin to tip over -- similar to an inverted pendulum -- when compensating for very small values of $r_z$. To avoid this behaviour, it became beneficial to only accept filter outputs where $\sigma$ is approximately less than $0.5r_z$. In practice, this means the vertical balancing procedure included two or three iterations before halting. 





\section{Passive Balancing Using Least-Squares Estimation}
% experimental results only use 1 RW
% simulation results should include 1 RW and 4 RWs to compare (unless the one RW setup turns out good)

The least-squares estimation method is simulated using the intended four reaction wheel setup for the SADS. The platform's state during a batch collection run is shown in \Cref{fig:LSR_sim_excitation}. The torque profile is designed to have the platform oscillate as close as possible to it's $\pm20^{\circ}$ travel limits for roll and pitch, while simultaneously not saturating the reaction wheels. Both of these goals are met in the example run shown.

\begin{figure}[ht]
  \centering
  \begin{subfigure}[t]{0.47\textwidth}
    \includegraphics[width=\linewidth]{plots/LSR_sim_all_runs.png}
    \caption{All 10 iterations}\label{fig:LSR_sim_runs_a}
  \end{subfigure}\hfill
  \begin{subfigure}[t]{0.47\textwidth}
    \includegraphics[width=\linewidth]{plots/LSR_sim_last_5_runs.png}
    \caption{Final 5 iterations}\label{fig:LSR_sim_runs_b}
  \end{subfigure}
  \caption{Simulated results of least-squares estimation method}
  \label{fig:LSR_sim_runs}
\end{figure}

The estimated value for $\bm{r}$ over the course 10 iterations is shown in \Cref{fig:LSR_sim_runs}. The estimated values rapidly converge during the first few iterations, and eventually begin to oscillatate about $\bm{r}=\bm{0}$.The oscillations for all three components of the estimate of $\bm{r}$ are bounded between $\pm3\times10^{-7}m$. The implications this has on the final balancing results are better demonstrated in \Cref{fig:LSR_sim_confidence}. The estimated value for $r_x$ is plotted directly against the true value of $r_x$ over many iterations. While the $r_x$ estimate oscillates about 0, the true value oscillates about a non-zero, constant value. This small error between the two is not measurable to the batch-estimation algorithm due to factors like sensor and process noise. 

To address this, the estimated values $r_x$, $r_y$, and $r_z$ after convergence are collected and the standard deviation $\sigma$ of the resulting sets are computed. In general, it was observed that greater variance in the converged values corresponded to greater error between the true and estimated values. In other words, the variance of the converged values (which are directly related to the amplitude of the oscillations) help place a bound on estimators performance. For example in \Cref{fig:LSR_sim_confidence}, $\sigma$ is used to create a confidence interval, and the true value of $r_x$ safely lies within the interval.

\begin{figure}[ht]
    \centering
    \includegraphics[width=\linewidth]{plots/LSR_sim_excitation}
    \caption{Simulated platform excitation during batch estimation}
    \label{fig:LSR_sim_excitation}
\end{figure}

\begin{figure}[ht]
    \centering
    \includegraphics[width=\linewidth]{plots/LSR_sim_confidence.pdf}
    \caption{Comparison between estimated and true values}
    \label{fig:LSR_sim_confidence}
\end{figure}

With the batch estimation algorithm verified using simulated values, the procedure is repeated on the SADS. The platform's state during a batch collection run is shown in \Cref{fig:LSR_hardware_excitation}. While the experimental setup include only one reaction wheel, the SADS' pyramidal configuration means the one wheel contributes angular momentum to the system in all three axes of the body frame. 

\begin{figure}[ht]
    \centering
    \includegraphics[width=\linewidth]{plots/LSR_hardware_excitation.png}
    \caption{Experimental platform excitation during batch estimation}
    \label{fig:LSR_hardware_excitation}
\end{figure}

\Cref{fig:LSR_hardware_runs} shows the estimated value of $r$ over 10 experimental iterations. $r$ again converges in a similar manner to the simulated test. It is important to note however the difference in orders of magnitude between \Cref{fig:LSR_sim_runs_b} and \Cref{fig:LSR_hardware_runs_b}. The oscillations after inital converge in the experimental test have a drastically higher amplitiude than the simulated case. As discussed previously, the implication of this is that the error in the experimental estimator is worse than the error in the simulated one. This difference is combination of both modeling errors in the simulation, but also a potential lack of excitation from the one reaction wheel used in the experimental setup. 


\begin{figure}[ht]
  \centering
  \begin{subfigure}[t]{0.47\textwidth}
    \includegraphics[width=\linewidth]{plots/LSR_hardware_all_runs.png}
    \caption{All 10 iterations}\label{fig:LSR_hardware_runs_a}
  \end{subfigure}\hfill
  \begin{subfigure}[t]{0.47\textwidth}
    \includegraphics[width=\linewidth]{plots/LSR_hardware_last_5_runs.png}
    \caption{Final 5 iterations}\label{fig:LSR_hardware_runs_b}
  \end{subfigure}
  \caption{Experimental results of least-squares estimation method}
  \label{fig:LSR_hardware_runs}
\end{figure}



\section{Active Balancing Using Adaptive Control}

\subsection{No Reaction Wheels}
% the experimental results for this are in a seperate repo - SADS Adaptive

\subsection{Full Reaction Wheel Setup - Simulink Only}

\begin{figure}[ht]
    \centering
    \includegraphics[width=\linewidth]{plots/three_axis_sim_excitation.png}
    \caption{Simulated platform excitation for 3-axis adaptive control}
\end{figure}

\begin{figure}[ht]
  \centering
  \begin{subfigure}[t]{0.47\textwidth}
    \includegraphics[width=\linewidth]{plots/three_axis_sim_positions_1.png}
    \caption{1st iteration}\label{fig:a}
  \end{subfigure}\hfill
  \begin{subfigure}[t]{0.47\textwidth}
    \includegraphics[width=\linewidth]{plots/three_axis_sim_positions_2.png}
    \caption{2nd iteration}\label{fig:b}
  \end{subfigure}
  \caption{Simulated sliding mass positions during 3-axis adaptive control}
  \label{fig:LSR_hardware_iterations}
\end{figure}

\subsection{Verification}

\begin{figure}[ht]
    \centering
    \includegraphics[width=\linewidth]{plots/hardware_verification_KE.pdf}
    \caption{Kinetic energy of the simulator before and after using two-step PID balancing}
\end{figure}

\begin{figure}[ht]
    \centering
    \includegraphics[width=\linewidth]{plots/hardware_verification_torque.pdf}
    \caption{Norm of torque acting on the simulator before and after using two-step PID balancing}
\end{figure}

\begin{table}[ht]
\caption{Comparison of Balancing Methods and Performance Metrics}\label{table:balancing_methods}
\centering
\renewcommand{\arraystretch}{1.3}

\begin{tabularx}{\textwidth}{
    >{\raggedright\arraybackslash}p{4cm}
    >{\centering\arraybackslash}p{3cm}
    >{\centering\arraybackslash}X
}
\toprule
\textbf{Method} & \textbf{Avg. Torque [N·m]} & \textbf{max ($\Delta$KE) [J]} \\
\midrule
Manual Balancing & 0.0652 & 0.0116 \\ \addlinespace[0.2em]
Results from Dam & 0.0244 & 0.00782 \\ \addlinespace[0.2em]
PID + UKF Simulated & 0.00321 & 0.000965 \\ \addlinespace[0.2em]
PID + UKF Experimental & 0.00166 & 0.000236 \\ \addlinespace[0.2em]
Batch Estimation Simulated & $3.89\times10^{-5}$ & 0.000116 \\ \addlinespace[0.2em]
Batch Estimation Experimental & 0.0146 & 0.00226 \\ \addlinespace[0.2em]
3-Axis Adaptive Simulated & 0.00299 & 0.000741 \\ \addlinespace[0.2em]
\bottomrule
\end{tabularx}
\end{table}

\iffalse
SIM RESULTS ------------------------------
Results PID:
horizontal: 2.5e-5
verit 6.7e-7
KE = 0.00096508
torque = 0.0032123

Results Batch:
horizontal: 1e-7
vert: 2.2e-7
KE = 0.00011609;
torque = 3.8857e-05;

Results UKF
r_z = 6.784e-07

Results 3-axis
horizontal: 4.8261e-06
vertical: 9.7241e-05
KE = 0.00074126
torque = 0.0029949

HARDWARE RESULTS ----------------------------------
Results PID:
KE = 0.00023616
torque =  0.0016608;

Results Passive:
KE = 0.0022634
torque = 0.014569

Manual Balancing:
KE = 0.011555
torque = 0.065169

Dam:
KE = 0.0078203
torque = 0.0244

\fi